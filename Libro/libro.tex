\documentclass{article}
% Preambolo
\usepackage{graphicx} % per inserimento immagini
\usepackage{amsmath} % per inserimento formule matematiche
\usepackage[italian]{babel}

\title{Titolo del documento}
\author{Taiwo Solomon}
\date{Lunedì 06 Febbraio 2023}

% corpo

\begin{document}
\maketitle

\begin{abstract}
    Abstract dell'articolo
\end{abstract}

\tableofcontents

\section{Sezione di introduzione}\label{sec:prova}
Ciao da LaTex

\section*{Conclusioni}
Un paragrafo di testo
un alro paragrafo\footnote{Contenuto della nota} che si riferisce alla sezione\ref{sec:prova} a pagina\pageref{sec:prova}

\section{Elenchi puntati e numerati}

\begin{itemize}
    \item primo elemento
    \item secondo elemento
\end{itemize}

\begin{enumerate}
    \item primo elemento
    \item secondo elemento
\end{enumerate}

\section{Allineamento paragrafi}

\begin{flushright}
    Testo allineato a destra
\end{flushright}

\begin{flushleft}
    Testo allineato a sinistra
\end{flushleft}

\begin{center}
    testo centrato
\end{center}

\section{Citazioni}

\begin{quote}
    Citazione dotta
\end{quote}

\section{Verbatim}

\begin{verbatim}
    print("Ciao Mondo");
\end{verbatim}

\section{Immagini}

\includegraphics[width=\textwidth]{prova.jpg}

\section{Matematica}

Sia $f(x)$ una funzione, allora l'integrale di f vale zero
$$\iint f(x)dx = 0$$

\begin{equation}
    \iint f(x)dx = 0\label{eq:test}
\end{equation}

Come si vede dall'equazione \ref{eq:test}

Grazie al prof. Andrea Pollini per il tutorial su LaTeX

\end{document}